\chapter{How Headings Should Be Formatted}

\section{Spacing For Different Headings}

I hope that this template will help you in formatting your dissertation, which can be one of the most painful and frustrating parts of the process. Hopefully the style file provided will take care of all formatting needs, but there will inevitably something you will need to fix by hand and so knowing the details will be helpful. The most important piece of information that you need to know for spacing issues is the length of a double and triple space. The graduate school defines these in terms Microsoft Word, but is of no help in \LaTeX. A double space is defined as .25in and a triple space is defined as .45in. Now that we have that down, let us talk about headings.

There can be five different heading levels in the dissertation. A level 1 heading spells out the chapter number in all capital letters and is automatically generated when a new chapter is defined. The headings in the pre-document stuff (List Of Figures, List Of Tables, etc.) are also level 1 headings. Level 1 headings need to have a 1.5in margin above it. A level 2 heading is reserved for the name of the chapter in title case. The chapter name needs to be \textbf{single spaced} if longer than one line. There should be a \textbf{double space} between the level 1 and level 2 heading.

\section{Level 3 Heading (Title Case)}
A level 3 heading signifies the start of a section in the chapter. You define a section heading with the \verb|\section{}| command. There needs to be a triple space between a level 2 and level 3 heading and a double space after. If you choose not to start the chapter with a section, there still needs to be a triple space between the level 2 heading and the text. For a level 3 heading within the chapter, there needs to be a triple space before the heading and a double space after.

\subsection{Level 4 Heading (Title Case)}

A level 4 heading is defined using the \verb|\subsection{}| command. Just as with level 3 headings there needs to be a triple space before and a double space after. The only difference between a level 3 and a level 4 heading is the positioning. Level 3 headings are centered while level 4 headings are left adjusted.

\subsubsection{Level 5 heading (Sentence case)}

A level 5 heading is defined using the \verb|\subsubsection{}| command. The level 5 heading needs to be in sentence case, is in the same line as the text, and is indented. Also, just like level 3 and 4 headings, there needs to be a triple space beforehand.


